\documentclass[12pt,a4paper]{article}

\usepackage[utf8]{inputenc}
\usepackage[T1]{fontenc}
\usepackage[polish]{babel}
\usepackage{url}
\usepackage{hyperref}

\usepackage{geometry}
\geometry{margin=2.5cm}
\usepackage{enumitem}
\setlist{nosep}
\setlist[enumerate,2]{label=\arabic*)}



\renewcommand{\thesection}{\S~\arabic{section}}

\begin{document}

\begin{center}
    {\LARGE \textbf{Regulamin}} \\[0.5cm]
    {\Large {Koło Naukowe Nanoelektronika i Mikroelektronika ,,NaMi''}} \\
    \vspace{0.5cm}
    {\Large \textbf{Wrocław, 2025}}
\end{center}

\newpage
\section{Postanowienia ogólne}
\begin{enumerate}
    \item Koło Naukowe Nanoelektronika i Mikroelektronika ,,NaMi'', jest uczelnianą organizacją studencką i doktorancką, w rozumieniu art. 111 ustawy z dnia 20 lipca 2018 r. – Prawo o szkolnictwie wyższym i nauce, zwaną dalej Kołem Naukowym.
    \item Koło Naukowe może używać skróconej nazwy w brzmieniu ,,KN NaMi''.
    \item Siedzibą Koła Naukowego jest Politechnika Wrocławska, bud. M-4, pok. 109. \\
    Adres do korespondencji: \\
    ul. Długa 61-65 \\
    53-633 Wrocław \\
    strona internetowa: \url{https://nami.pwr.edu.pl} \\
    e-mail: \href{mailto:kn.nami@pwr.edu.pl}{kn.nami@pwr.edu.pl}

    \item Koło Naukowe opiera swoją działalność na dobrowolnej pracy ogółu członków wraz z jego władzami.
    \item Koło Naukowe działa w ramach Wydziału Elektroniki, Fotoniki i Mikrosystemów Politechniki Wrocławskiej.
    \item Podstawy prawne działania Koła Naukowego stanowią:
    \begin{enumerate}
        \item regulamin Koła Naukowego, zwany dalej ,,Regulaminem'',
        \item akty prawa wewnętrznego obowiązujące w Politechnice Wrocławskiej,
        \item przepisy prawa powszechnie obowiązującego, w tym ustawa o szkolnictwie wyższym i nauce.
    \end{enumerate}
    \item Czas działania Koła Naukowego jest nieokreślony.
    \item Rektor Politechniki Wrocławskiej uchyla akt organu Koła Naukowego niezgodny z przepisami prawa powszechnie obowiązującego, Statutem Uczelni lub aktami prawa wewnętrznego obowiązującymi w Politechnice Wrocławskiej.
\end{enumerate}

\section{Cele Koła Naukowego}
Celami Koła Naukowego są:
\begin{enumerate}
    \item prowadzenie działalności naukowo-badawczej w obszarze nanoelektroniki i mikroelektroniki,
    \item rozwijanie wiedzy i umiejętności studentów oraz doktorantów w zakresie technologii półprzewodnikowych, nanotechnologii oraz systemów mikroelektronicznych,
    \item tworzenie warunków sprzyjających pracy zespołowej i realizacji projektów badawczych oraz konstrukcyjnych,
    \item uczestnictwo w krajowych i międzynarodowych konferencjach, warsztatach i szkoleniach związanych z tematyką Koła Naukowego,
    \item popularyzacja nauki oraz promocja osiągnięć członków Koła Naukowego w środowisku akademickim i poza nim,
    \item współpraca z innymi kołami naukowymi, instytucjami badawczymi oraz partnerami przemysłowymi,
    \item integracja środowiska studenckiego i doktoranckiego wokół zagadnień związanych z tematyką Koła Naukowego.
\end{enumerate}

\newpage

\section{Realizacja celów Koła Naukowego}
Koło Naukowe realizuje cele, o których mowa w \S 2 poprzez:
\begin{enumerate}
    \item prowadzenie projektów naukowo-badawczych i konstrukcyjnych,
    \item organizowanie seminariów, warsztatów i szkoleń,
    \item udział w konferencjach i konkursach naukowych,
    \item współpracę z innymi organizacjami studenckimi oraz partnerami zewnętrznymi.
\end{enumerate}

\section{Źródła finansowania}
Do realizacji celów, o których mowa w \S 2, Koło Naukowe może korzystać z źródeł finansowania w szczególności pochodzących z:
\begin{enumerate}
    \item środków Politechniki Wrocławskiej pozyskanych zgodnie z Porozumieniem w sprawie finansowania działalności studentów i doktorantów w Politechnice Wrocławskiej,
\end{enumerate}


\section{Postanowienia końcowe}
\begin{enumerate}
    \item Zmiana Regulaminu wymaga uchwały Walnego Zgromadzenia Członków Koła Naukowego podjętej większością głosów przsy obecności co najmniej połowy liczby członków Koła Naukowego.
    \item Zarzą Koła Naukowego ma obowiązek pisemnego zawiadamiania Rektora Politechniki Wrocławskiej, za pośrednictwem właściwej jednostki administracyjnej, o każdej zmianie treści niniejszego Regulaminu, zmianach w organach Koła, zmianie Opiekuna oraz każdej uchwale Walnego Zgromadzenia Członków, w terminie 14 dni od dnia dokonania zmiany. 
    \item Walne Zgromadzenie Członków Koła Naukowego większością 2/3 głosów przy obecności co najmniej połowy liczby członków Koła Naukowego może podjąć uchwały w sprawie zawieszenia działalności lub rozwiązania Koła Naukowego.
    \item Rektor Politechniki Wrocławskiej, w drodze decyzji administracyjnej, rozwiązuje uczelnianą organizację studencką i doktorancką, w tym Koło Naukowe, która rażąco lub uporczywie narusza przepisy prawa powszechnie obowiązującego, statutu uczelni, regulaminu studiów lub regulaminu Koła Naukowego.
    \item Koło Naukowe ulega rozwiązaniu na mocy niniejszego Regulaminu, jeżeli w ciągu dwóch lat nie odbyło się Walne Zgromadzenie Członków Koła Naukowego, którego protokół przekazano Rektorowi Politechniki Wrocławskiej, za pośrednictwem właściwej jednostki administracyjnej.
\end{enumerate}

\end{document}
